
\usepackage[utf8]{inputenc}  
\usepackage[T1]{fontenc}
\usepackage{geometry}
\usepackage{listings}
\usepackage{url}
\usepackage{hhline}
\usepackage[normalem]{ulem}
\usepackage[font={color=foreground,bf}]{caption}
\usepackage{array,booktabs,arydshln}
\usepackage{pdflscape}
\usepackage{xcolor}
\usepackage{inconsolata}
\usepackage{pgfplots}
\usepackage{listings}
\pgfplotsset{compat=1.17}
\usetikzlibrary{shadows}

\def\imgBaseDir{img}
\def\imgSubDir{}
\newcommand\setimgdir[1]{\def\imgSubDir{#1/}}

\newcommand\img[1]{\includegraphics[width=\linewidth]{\imgBaseDir/\imgSubDir#1}}

\newcommand\darktheme {
	
	\colorlet{foreground}{black!25}
	\colorlet{background}{black!74}
	\colorlet{focus}{yellow!50}
	\definecolor{codeColor}{RGB}{85,255,85}
	\colorlet{codeBackground}{black!80}
	\def \imgBaseDir {imgDark}
	
	
	
}
%\rowcolors{1}{background}{background!95} % Alternated array colors
%%%%%%%%%%%%%%%%%%%%%%%%%%%%%%%%%%%%%%%%%%%%%%%%%%%%%%%   BASIC STYLES   %%%%%%%%%%%%%%%%%%%%%%%%%%%%%%%%%%%%%%%%%%%%%%%%%%%%%%%
\colorlet{foreground}{black}
\colorlet{background}{white}
\colorlet{focus}{blue}
%\definecolor{codeColor}{RGB}{0,0,0}
\colorlet{codeBackground}{black!10}



\renewcommand\emph[1]	{{ \color{red!40}	\textbf{#1}} 			}
\newcommand\mynote[1]	{{ \color{focus!50} \textbf{NOTE:	} #1 }\\}
\newcommand\warning[1]	{{ \color{focus!70} \textbf{WARNING:} #1 }\\}
\newcommand\todo[1]		{{ \color{focus} 	\textbf{TODO:} 	  #1 }\\}
%%%%%%%%%%%%%%%%%%%%%%%%%%%%%%%%%%%%%%%%%%%%%%%%%%%%%%%   CODE   %%%%%%%%%%%%%%%%%%%%%%%%%%%%%%%%%%%%%%%%%%%%%%%%%%%%%%%
\lstset{
	firstnumber=1,
	texcl=true,inputencoding=latin1,
	backgroundcolor=\color{codeBackground},   % choose the background color; you must add \usepackage{color} or \usepackage{xcolor}; should come as last argument,
	basicstyle=\footnotesize\ttfamily,      % the size of the fonts that are used for the code
	breakatwhitespace=true,         % sets if automatic breaks should only happen at whitespace
	breaklines=true,
	captionpos=b,                    % sets the caption-position to bottom
	commentstyle=\color{green!65!black},    % comment style
	deletekeywords={...},            % if you want to delete keywords from the given language
	escapeinside={\%*}{*)},          % if you want to add LaTeX within your code
	extendedchars=true,              % lets you use non-ASCII characters; for 8-bits encodings only, does not work with UTF-8
	frame=single,	                   % adds a frame around the code
	keepspaces=true,                 % keeps spaces in text, useful for keeping indentation of code (possibly needs columns=flexible)
	keywordstyle=\color{blue!60},       % keyword style
	language=C,                 % the language of the code,
	morekeywords={*,...},            % if you want to add more keywords to the set
	numbers=left,                    % where to put the line-numbers; possible values are (none, left, right)
	numbersep=5pt,                   % how far the line-numbers are from the code
	numberstyle=\tiny\color{black!40}, % the style that is used for the line-numbers
	rulecolor=\color{black},         % if not set, the frame-color may be changed on line-breaks within not-black text (e.g. comments (green here))
	showspaces=false,                % show spaces everywhere adding particular underscores; it overrides 'showstringspaces'
	showstringspaces=false,          % underline spaces within strings only
	showtabs=false,                  % show tabs within strings adding particular underscores
	stepnumber=5,                    % the step between two line-numbers. If it's 1, each line will be numbered
	stringstyle=\color{violet!50},     % string literal style
	tabsize=4,	                   % sets default tabsize to 2 spaces
	title=\lstname                   % show the filename of files included with \lstinputlisting; also try caption instead of title	
}

\newcommand\icode[1]{\lstinline[breaklines=true,breakatwhitespace=false, basicstyle=\ttfamily\small]{#1}}\lstnewenvironment{code}{}{}
\lstnewenvironment{bash}{\lstset{language=Bash}}{}
\lstnewenvironment{fcode}[1]{\lstset{caption=#1}}{}

% % Environment for list in block most common case
\newenvironment{bl}[1][] % By default, no title
	{\begin{block}{#1}\begin{itemize}}
	{\end{itemize}\end{block}}


\newcolumntype{Y}{>{\centering\arraybackslash}X}
